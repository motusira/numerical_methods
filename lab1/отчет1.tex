\documentclass{article}

\usepackage[utf8]{inputenc}
\usepackage[T2A]{fontenc} 
\usepackage[russian]{babel}

\usepackage{enumitem}
\usepackage{indentfirst}
\usepackage{amsmath}

\usepackage[left=2cm, right=2cm, top=2cm, bottom=2cm]{geometry}


\begin{document}

\begin{titlepage}

    \centering
    
    {\large Санкт-Петербургский Политехнический Университет Петра Великого \\
					Физико-Механический Институт \\
					Высшая школа прикладной математики и вычислительной физики\par}
					
	\vspace{1cm}
    
    {\LARGE Лабораторная работа 2.1 \\ 
    По дисциплине «Численные методы» \\ 
    На тему «Приближение табличных функций»\par}
    
    \vspace{1.5cm} 
    
     \vfill
    
	\begin{center}
        \begin{tabbing}
            Выполнил студент группы 5030102/30003: \= \kill
            Выполнил студент группы 5030102/30003: \> \hspace{8cm}Черницын И. А. \\
            Преподаватель: \> \hspace{8cm}Добрецова С. Б.
        \end{tabbing}
    \end{center}
    
    \vspace{1cm}
    
    {\large Санкт-Петербург\par}
    
    {\large 2025 год\par}
    
\end{titlepage}

\tableofcontents

\section{Постановка задачи}

В данной работе содержатся результаты исследования методов приближения табличных функций с использованием интерполяционных полиномов, а именно --- полинома Ньютона, построенного «слева-направо» с использованием Чебышевской сетки. Исследование будет проводиться на примере вычисления значений полином для приближения функции в заданных точка.

В качестве результатов исследования будут получены графики:
\begin{itemize}[label=$\cdot$]
    \item функции
    \item полинома Ньютона с отмеченными узлами
\end{itemize}
а также получена зависимость фактической ошибки на отрезка для отмеченных узлов и установлена зависимость максимальной фактической ошибки от числа узлов.

\section{Описание метода}

Интерполяционный полином для функции $y \in C[a, b]$ по узлам
$x_1, x_2, ... ,\allowbreak x_n$ можно представить формулой Ньютона:

\begin{equation}
\begin{split}
P_n(x) &= y(x_0) + (x - x_0) y(x_0, x_1) + (x - x_0)(x - x_1) y(x_0, x_1, x_2) + \dots \\
& + (x - x_0)(x - x_1) \dots (x - x_{n-1}) y(x_0, x_1, \dots, x_n) = \\
& = \sum_{i=0}^{n} (y(x_0, x_1, \dots, x_i) \prod_{k=0}^{i-1} (x - x_k)).
\end{split}
\end{equation}

В данной формуле под $y(x_0,x_1), y(x_0,x_1,x_2,)$ и т.д. подразумевают разделенные разности, например:

\begin{equation}
y(x_0,x_1) = \frac{y(x_1) - y(x_0)}{x_1-x_0}
\end{equation} --- разделенная разность 1-го порядка,

\begin{equation}
y(x_0,x_1,x_2) = \frac{y(x_1,x_2) - y(x_0,x_1)}{x_2-x_0}
\end{equation} --- разделенная разность 2-го порядка,

\begin{equation}
y(x_0,x_1,\dots,x_i) = \frac{y(x_1,x_2,\dots,x_i) - y(x_0,x_1,\dots,x_{i-1})}{x_k-x_0}
\end{equation} --- разделенная разность i-го порядка.

Для того, чтобы полином был единственным, его степень должна быть на
единицу меньше количества точек, и все точки должны быть попарно различны.

\section{Тестовый пример}

\section{Результаты исследования}

\section{Выводы}

\end{document}

