\documentclass{article}

\usepackage[utf8]{inputenc}
\usepackage[T2A]{fontenc} 
\usepackage[russian]{babel}

\usepackage{enumitem}
\usepackage{indentfirst}
\usepackage{amsmath}

\usepackage{float}
\usepackage{pgfplots}
\pgfplotsset{width=0.95\textwidth, compat=1.18}

\usepackage[left=2cm, right=2cm, top=2cm, bottom=2cm]{geometry}


\begin{document}

\begin{titlepage}

    \centering
    
    {\large Санкт-Петербургский Политехнический Университет Петра Великого \\
					Физико-Механический Институт \\
					Высшая школа прикладной математики и вычислительной физики\par}
					
	\vspace{1cm}
    
    {\LARGE Лабораторная работа 2.1 \\ 
    По дисциплине «Численные методы» \\ 
    На тему «Приближение табличных функций»\par}
    
    \vspace{1.5cm} 
    
     \vfill
    
	\begin{center}
        \begin{tabbing}
            Выполнил студент группы 5030102/30003: \= \kill
            Выполнил студент группы 5030102/30003: \> \hspace{8cm}Черницын И. А. \\
            Преподаватель: \> \hspace{8cm}Добрецова С. Б.
        \end{tabbing}
    \end{center}
    
    \vspace{1cm}
    
    {\large Санкт-Петербург\par}
    
    {\large 2025 год\par}
    
\end{titlepage}

\tableofcontents

\section{Постановка задачи}

В данной работе содержатся результаты исследования методов приближения табличных функций с использованием интерполяционных полиномов, а именно --- полинома Ньютона, построенного «слева-направо» с использованием Чебышевской сетки. Исследование будет проводиться на примере вычисления значений полином для приближения функции в заданных точка.

В качестве результатов исследования будут получены графики:

\begin{itemize}
    \item функции
    \item полинома Ньютона с отмеченными узлами
\end{itemize}

а также будет получена зависимость фактической ошибки на отрезке для отмеченных узлов и установлена зависимость максимальной фактической ошибки от числа узлов.

\section{Описание метода}

Интерполяционный полином для функции $y \in C[a, b]$ по узлам
$x_1, x_2, ... ,\allowbreak x_n$ можно представить формулой Ньютона:

\begin{equation}
\begin{split}
P_n(x) &= y(x_0) + (x - x_0) y(x_0, x_1) + (x - x_0)(x - x_1) y(x_0, x_1, x_2) + \dots \\
& + (x - x_0)(x - x_1) \dots (x - x_{n-1}) y(x_0, x_1, \dots, x_n) = \\
& = \sum_{i=0}^{n} (y(x_0, x_1, \dots, x_i) \prod_{k=0}^{i-1} (x - x_k))
\end{split}
\end{equation}

В данной формуле под $y(x_0,x_1), y(x_0,x_1,x_2,)$ и т.д. подразумевают разделенные разности, например:

\begin{equation}
y(x_0,x_1) = \frac{y(x_1) - y(x_0)}{x_1-x_0}
\end{equation} --- разделенная разность 1-го порядка,

\begin{equation}
y(x_0,x_1,x_2) = \frac{y(x_1,x_2) - y(x_0,x_1)}{x_2-x_0}
\end{equation} --- разделенная разность 2-го порядка,

\begin{equation}
y(x_0,x_1,\dots,x_i) = \frac{y(x_1,x_2,\dots,x_i) - y(x_0,x_1,\dots,x_{i-1})}{x_k-x_0}
\end{equation} --- разделенная разность i-го порядка.

\bigskip

Для построени интерполяционного многочлена Ньютона с использованием сетки Чебышева использовался следующий алгоритм:

\begin{enumerate}
    \item построение сетки Чебышева для выбранного промежутка с заданным числом узлов
    \item вычисление разделенных разностей
    \item подстановка полученных значений в приведенную выше формулу
\end{enumerate}

\bigskip

Также стоит отметить что, чтобы полином был единственным, его степень должна быть на
единицу меньше количества точек, и все точки должны быть попарно различны.

\section{Тестовый пример}

Для иллюстрации работы метода рассмотрим принцип его работы для функции:

\begin{equation}
y(x) = 3x - ln x + 3 cos x
\end{equation}

на промежетке $[1, 3]$, использовав для приближения три узла.
Сначала построим сетку Чебышева:

\begin{gather}
t_k \in [-1, 1], \qquad t_k = cos \frac{\pi (2k + 1)}{2(n+1)}, \qquad k = \overline{0, 2} \nonumber \\
t_0 = cos \frac{\pi (2 \cdot 0 + 1)}{2(2 + 1)} = cos \frac{\pi}{6} \approx 0.866 \nonumber \\
t_1 = cos \frac{\pi (2 \cdot 1 + 1)}{2(2 + 1)} = cos \frac{3\pi}{6} = 0 \nonumber \\
t_2 = cos \frac{\pi (2 \cdot 2 + 1)}{2(2 + 1)} = cos \frac{5\pi}{6} \approx -0.866
\end{gather}

\begin{gather}
x_k \in [1, 3], \qquad x_k = \frac{1 + 3}{2} + \frac{3 - 1}{2} t_k \nonumber \\
x_0 = \frac{4}{2} + \frac{2}{2} t_0 = 2 + 0.866 = 2.866 \nonumber \\
x_1 = 2 + 0 = 2.866 \nonumber \\
x_2 = 2 - 0.866 = 1.134
\end{gather}

Теперь вычислим значения исходной функции в данных точках, после чего сможем построить многочлен Ньютона:

\begin{gather}
y(x_0) = 3 * 2.866 - ln 2.866 + 3 cos 2.866 = 4.658 \nonumber \\
y(x_1) = 4.058 \nonumber \\
y(x_2) = 4.545
\end{gather}

В результате имеем следующий набор точек:

\begin{equation}
(2.866, 4.658), \quad (2, 4.058), \quad (1.134, 4.545)
\end{equation}

Вычислим разделенные разности:

\begin{gather}
y(x_0, x_1) = \frac{y(x_1) - y(x_0)}{x_1 - x_2} = \frac{4.058 - 4.658}{2 - 2.866} \approx 0.693 \nonumber \\
y(x_1, x_2) = \frac{4.545 - 4.058}{1.134 - 2} \approx -0.562 \nonumber \\
y(x_0, x_1, x_2) = \frac{y(x_1, x_2) - y(x_0, x_1)}{x_2 - x_0} = \frac{-0.562 - 0.693}{1.134 - 2.886} \approx 0.725
\end{gather}

Получим полином:

\begin{align}
P_3(x) &= y(x_0) + (x - x_0)y(x_0, x_1) + (x - x_0)(x - x_1)y(x_0, x_1, x_2) = \nonumber \\
        &= 4.658 + 0.693(x - 2.866) + 0.725(x - 2.866)(x - 2) \approx \nonumber \\
        &\approx 0.725x^2 - 2.835x + 6.827
\end{align}

Проверим значения в узлах сетки:

\begin{gather}
P_3(2.866) \approx 4.657 \nonumber \\
P_3(2) \approx 4.057 \nonumber \\
P_3(1.134) \approx 4.544
\end{gather}

Как видно из полученных результатов, мы получили приближение функции интерполяционным полиномом Ньютона. В узлах сетки расхождение со значениями функции должно быть равно 0, но из-за накопления погрешности при округлении мы получили небольшое расхождение (порядка 0.001).

\section{Результаты исследования}

Исследовать метод интерполяции Ньютона будем на примере следующей функции:

\begin{equation}
y(x) = x^2 + 1 - arccos(x)
\end{equation}

Для построения полинома будем использовать сетку Чебышева с 10 узлами. Строить графики будем на промежутке $[-1, 1]$ с шагом 0.1, а также будем устанавливать фактическую ошибку в каждой точке построения

\begin{figure}[H]
\centering
% График функции и полинома
\begin{tikzpicture}
\begin{axis}[
    xlabel={$x$},
    ylabel={$y$},
    grid=major,
legend pos=north west,
    %axis lines=middle,
    axis line style={line width=1.5pt},
    domain=-0.6:0.2,       % Область определения для функции
    samples=100           % Количество точек расчета
]
% Исходная функция (аналитическое задание)
\addplot[
    blue,
    thick,
    mark=none
] {x^2 + 1 - rad(acos(x))};
\addlegendentry{$y = x^2 + 1 - \ln(x)$}

% Полином
\addplot[
	red,
	thick,
	dashed,
	mark=*,
    mark size=2pt,
    mark options={fill=white}
] table[x index=0, y index=1] {
0.192314 -0.340292
0.132588 -0.420237 
0.022228  -0.548072
-0.121964 -0.678189 
-0.278036  -0.775241 
-0.422228  -0.828422
-0.532588  -0.848802
-0.592314  -0.853888
};
\addlegendentry{Полином}
\end{axis}
\end{tikzpicture}
\caption{сравнение функции и полинома}
\end{figure}

\begin{figure}[H]
\centering
% График ошибки
\begin{tikzpicture}
\begin{axis}[
    xlabel={$x$},
    ylabel={Ошибка},
    grid=major,
    tick style={line width=1pt},
    axis line style={line width=1.5pt},
]
\addplot[black, thick, mark=*, mark size=1.5pt] table[x index=0, y index=1] {
0.192314  0.000000
0.132588  0.000000
0.022228  0.000000
-0.121964  0.000000
-0.278036 0.000000
-0.422228  0.000000
-0.532588  0.000000
-0.592314  0.000000
};
\end{axis}
\end{tikzpicture}
\caption{фактическая ошибка для точки}

\end{figure}

Как видно из полученных графиков, значение полинома полностью совпадает с значением фукнции в узлах сетки.

\bigskip

Теперь установим зависимость фактической ошибки от числа узлов. Фактическая ошибка будет устанавливаться в точках, не являющихся узлами интерполяции. Число узлов будет изменяться от 2 до 70.

\begin{figure}[H]
\centering
% График ошибки
\begin{tikzpicture}
\begin{axis}[
    xlabel={Число узлов},
    ylabel={Ошибка},
    grid=major,
    ymode=log,
    extra y ticks={0},
    axis line style={line width=1.5pt},
]
\addplot[black, thick, mark=*, mark size=1.5pt] table[x index=0, y index=1] {
2 2.705189e-02
3 8.762599e-04
4 8.717340e-05
5 1.121642e-05
6 1.437674e-06
7 1.975247e-07
8 2.767908e-08
9 3.989559e-09
10 5.846718e-10
11 8.704704e-11
12 1.311640e-11
13 1.997402e-12
14 3.066436e-13
15 4.746203e-14
16 7.438494e-15
17 1.332268e-15
18 5.551115e-16
19 3.885781e-16
20 3.330669e-16
21 4.440892e-16
22 4.440892e-16
23 3.885781e-16
24 3.330669e-16
25 4.440892e-16
26 3.330669e-16
27 4.440892e-16
28 3.885781e-16
29 4.440892e-16
30 3.330669e-16
31 4.440892e-16
32 3.330669e-16
33 3.885781e-16
34 4.996004e-16
35 4.996004e-16
36 3.330669e-16
37 4.440892e-16
38 6.661338e-16
39 6.661338e-16
40 1.021405e-14
41 1.398881e-14
42 8.515411e-14
43 1.022515e-13
44 5.031531e-13
45 2.772227e-12
46 1.470379e-12
47 3.023803e-12
48 7.720302e-11
49 5.212294e-10
50 1.107480e-09
51 2.510001e-09
52 2.626324e-09
53 4.629975e-09
54 8.421127e-08
55 3.253141e-07
56 2.410680e-08
57 4.000437e-06
58 4.486381e-06
59 1.377306e-05
60 8.685170e-05
61 3.795232e-04
62 6.597166e-04
63 2.445475e-04
64 1.490118e-02
65 1.779749e-02
66 2.909063e-02
67 8.398810e-02
68 2.873879e-01
69 1.686540e+00
70 9.025528e+00
};
\end{axis}
\end{tikzpicture}
\caption{зависимость фактической ошибки от числа узлов}
\end{figure}

Как видно из полученного графика, фактическая ошибка имеет обратную степенную зависимость от числа узлов, и с увеличением числа узлов точность сначала резко возрастает, после чего доходит до предела, и при дальнейшем увеличении числа узлов точность начинает падать, поскольку полином обретает большую степень.

\section{Выводы}

В ходе лабораторной работы было проведено исследование одного из методов интерполяции - полинома Ньютона. Были построены графики исходной функции и полученного полинома, ошибки в каждой точке построения, а также была исследована зависимость фактической ошибики от числа узлов и построен соответствующий график.

\end{document}

