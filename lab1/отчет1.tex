\documentclass{article}

\usepackage[utf8]{inputenc}
\usepackage[T2A]{fontenc} 
\usepackage[russian]{babel}

\usepackage{enumitem}
\usepackage{indentfirst}
\usepackage{amsmath}

\usepackage{float}
\usepackage{pgfplots}
\pgfplotsset{width=0.95\textwidth, compat=1.18}

\usepackage[left=2cm, right=2cm, top=2cm, bottom=2cm]{geometry}


\begin{document}

\begin{titlepage}

    \centering
    
    {\large Санкт-Петербургский Политехнический Университет Петра Великого \\
					Физико-Механический Институт \\
					Высшая школа прикладной математики и вычислительной физики\par}
					
	\vspace{1cm}
    
    {\LARGE Лабораторная работа 2.1 \\ 
    По дисциплине «Численные методы» \\ 
    На тему «Приближение табличных функций»\par}
    
    \vspace{1.5cm} 
    
     \vfill
    
	\begin{center}
        \begin{tabbing}
            Выполнил студент группы 5030102/30003: \= \kill
            Выполнил студент группы 5030102/30003: \> \hspace{8cm}Черницын И. А. \\
            Преподаватель: \> \hspace{8cm}Добрецова С. Б.
        \end{tabbing}
    \end{center}
    
    \vspace{1cm}
    
    {\large Санкт-Петербург\par}
    
    {\large 2025 год\par}
    
\end{titlepage}

\tableofcontents

\section{Постановка задачи}

В данной работе содержатся результаты исследования методов приближения табличных функций с использованием интерполяционных полиномов, а именно --- полинома Ньютона, построенного «слева-направо» с использованием Чебышевской сетки. Исследование будет проводиться на примере вычисления значений полином для приближения функции в заданных точка.

В качестве результатов исследования будут получены графики:

\begin{itemize}[label=$\cdot$]
    \item функции
    \item полинома Ньютона с отмеченными узлами
\end{itemize}

а также будет получена зависимость фактической ошибки на отрезке для отмеченных узлов и установлена зависимость максимальной фактической ошибки от числа узлов.

\section{Описание метода}

Интерполяционный полином для функции $y \in C[a, b]$ по узлам
$x_1, x_2, ... ,\allowbreak x_n$ можно представить формулой Ньютона:

\begin{equation}
\begin{split}
P_n(x) &= y(x_0) + (x - x_0) y(x_0, x_1) + (x - x_0)(x - x_1) y(x_0, x_1, x_2) + \dots \\
& + (x - x_0)(x - x_1) \dots (x - x_{n-1}) y(x_0, x_1, \dots, x_n) = \\
& = \sum_{i=0}^{n} (y(x_0, x_1, \dots, x_i) \prod_{k=0}^{i-1} (x - x_k))
\end{split}
\end{equation}

В данной формуле под $y(x_0,x_1), y(x_0,x_1,x_2,)$ и т.д. подразумевают разделенные разности, например:

\begin{equation}
y(x_0,x_1) = \frac{y(x_1) - y(x_0)}{x_1-x_0}
\end{equation} --- разделенная разность 1-го порядка,

\begin{equation}
y(x_0,x_1,x_2) = \frac{y(x_1,x_2) - y(x_0,x_1)}{x_2-x_0}
\end{equation} --- разделенная разность 2-го порядка,

\begin{equation}
y(x_0,x_1,\dots,x_i) = \frac{y(x_1,x_2,\dots,x_i) - y(x_0,x_1,\dots,x_{i-1})}{x_k-x_0}
\end{equation} --- разделенная разность i-го порядка.

Для того, чтобы полином был единственным, его степень должна быть на
единицу меньше количества точек, и все точки должны быть попарно различны.

\section{Тестовый пример}

Для иллюстрации работы метода рассмотрим принцип его работы для функции:

\begin{equation}
y(x) = 3x - ln x + 3 cos x
\end{equation}

на промежетке $[1, 3]$, использовав для приближения три узла.
Сначала построим сетку Чебышева:

\begin{gather}
t_k \in [-1, 1], \qquad t_k = cos \frac{\pi (2k + 1)}{2(n+1)}, \qquad k = \overline{0, 2} \nonumber \\
t_0 = cos \frac{\pi (2 \cdot 0 + 1)}{2(2 + 1)} = cos \frac{\pi}{6} \approx 0.866 \nonumber \\
t_1 = cos \frac{\pi (2 \cdot 1 + 1)}{2(2 + 1)} = cos \frac{3\pi}{6} = 0 \nonumber \\
t_2 = cos \frac{\pi (2 \cdot 2 + 1)}{2(2 + 1)} = cos \frac{5\pi}{6} \approx -0.866
\end{gather}

\begin{gather}
x_k \in [1, 3], \qquad x_k = \frac{1 + 3}{2} + \frac{3 - 1}{2} t_k \nonumber \\
x_0 = \frac{4}{2} + \frac{2}{2} t_0 = 2 + 0.866 = 2.866 \nonumber \\
x_1 = 2 + 0 = 2.866 \nonumber \\
x_2 = 2 - 0.866 = 1.134
\end{gather}

Теперь вычислим значения исходной функции в данных точках, после чего сможем построить многочлен Ньютона:

\begin{gather}
y(x_0) = 3 * 2.866 - ln 2.866 + 3 cos 2.866 = 4.658 \nonumber \\
y(x_1) = 4.058 \nonumber \\
y(x_2) = 4.545
\end{gather}

В результате имеем следующий набор точек:

\begin{equation}
(2.866, 4.658), \quad (2, 4.058), \quad (1.134, 4.545)
\end{equation}

Вычислим разделенные разности:

\begin{gather}
y(x_0, x_1) = \frac{y(x_1) - y(x_0)}{x_1 - x_2} = \frac{4.058 - 4.658}{2 - 2.866} \approx 0.693 \nonumber \\
y(x_1, x_2) = \frac{4.545 - 4.058}{1.134 - 2} \approx -0.562 \nonumber \\
y(x_0, x_1, x_2) = \frac{y(x_1, x_2) - y(x_0, x_1)}{x_2 - x_0} = \frac{-0.562 - 0.693}{1.134 - 2.886} \approx 0.725
\end{gather}

Получим полином:

\begin{align}
P_3(x) &= y(x_0) + (x - x_0)y(x_0, x_1) + (x - x_0)(x - x_1)y(x_0, x_1, x_2) = \nonumber \\
        &= 4.658 + 0.693(x - 2.866) + 0.725(x - 2.866)(x - 2) \approx \nonumber \\
        &\approx 0.725x^2 - 2.835x + 6.827
\end{align}

Проверим значения в узлах сетки:

\begin{gather}
P_3(2.866) \approx 4.657 \nonumber \\
P_3(2) \approx 4.057 \nonumber \\
P_3(1.134) \approx 4.544
\end{gather}

Как видно из полученных результатов, мы получили приближение функции интерполяционным полиномом Ньютона. В узлах сетки расхождение со значениями функции должно быть равно 0, но из-за накопления погрешности при округлении мы получили небольшое расхождение (порядка 0.001).

\section{Результаты исследования}

Исследовать метод интерполяции Ньютона будем на примере следующей функции:

\begin{equation}
y(x) = x^2 + 1 - arccos(x)
\end{equation}

Для построения полинома будем использовать сетку Чебышева с 10 узлами. Строить графики будем на промежутке $[-1, 1]$ с шагом 0.1, а также будем устанавливать фактическую ошибку в каждой точке построения

\begin{figure}[H]
\centering
% График функции и полинома
\begin{tikzpicture}
\begin{axis}[
    %title={Сравнение функции и полинома},
    xlabel={$x$},
    ylabel={$y$},
    grid=major,
    legend pos=north west,
    axis lines=middle,
    axis line style={line width=1.5pt},
    xmin=-1.5, xmax=1.5,
]
% Исходная функция
\addplot[
	blue,
	thick,
mark=none
] table[x index=0, y index=1] {
-1.000000 -1.141593
-0.900000 -0.880566 
-0.800000 -0.858092 
-0.700000 -0.856194 
-0.600000 -0.854297 
-0.500000 -0.844395 
-0.400000 -0.822313
-0.300000 -0.785489
-0.200000 -0.732154 
-0.100000 -0.660964 
-0.000000 -0.570796 
0.100000 -0.460629 
0.200000 -0.329438 
0.300000 -0.176104 
0.400000 0.000721 
0.500000 0.202802 
0.600000 0.432705 
0.700000 0.694601 
0.800000 0.996499 
0.900000 1.358973 
1.000000 2.000000
};
\addlegendentry{Функция}

% Полином
\addplot[
	red,
	thick,
	dashed,
	mark=*,
    mark size=2pt,
    mark options={fill=white}
] table[x index=0, y index=1] {
-1.000000  -1.041998
-0.900000  -0.882183
-0.800000  -0.853789
-0.700000  -0.856478
-0.600000  -0.856282
-0.500000  -0.845123
-0.400000  -0.821721
-0.300000  -0.784757
-0.200000  -0.731954
-0.100000  -0.661086
-0.000000  -0.570796
0.100000  -0.460507
0.200000  -0.329638
0.300000  -0.176835
0.400000  0.000128
0.500000  0.203530
0.600000  0.434689
0.700000  0.694886
0.800000  0.992197
0.900000  1.360591
1.000000  1.900405
};
\addlegendentry{Полином}
\end{axis}
\end{tikzpicture}
\caption{сравнение функции и полинома}
\end{figure}

\begin{figure}[H]
\centering
% График ошибки
\begin{tikzpicture}
\begin{axis}[
    xlabel={$x$},
    ylabel={Ошибка},
    grid=major,
    extra y ticks={0},
    tick style={line width=1pt},
    axis line style={line width=1.5pt},
]
\addplot[black, thick, mark=*, mark size=1.5pt] table[x index=0, y index=1] {
-1.000000 0.099595
-0.900000  0.001617
-0.800000  0.004302
-0.700000  0.000284
-0.600000  0.001985
-0.500000  0.000728
-0.400000  0.000592
-0.300000  0.000732
-0.200000  0.000200
-0.100000  0.000122
-0.000000  0.000000
0.100000  0.000122
0.200000  0.000200
0.300000  0.000732
0.400000  0.000592
0.500000  0.000728
0.600000  0.001985
0.700000  0.000284
0.800000  0.004302
0.900000  0.001617
1.000000  0.099594
};
\end{axis}
\end{tikzpicture}
\caption{фактическая ошибка для точки}

\end{figure}

Как видно из полученных графиков, наиболее точное приближение мы получаем ближе к центру рассматриваемого отрезка, а при приближении к его границам точность быстро падает.

\bigskip

Теперь установим зависимость фактической ошибки от числа узлов. Фактическая ошибка будет устанавливаться в точках, не являющихся узлами интерполяции. Число узлов будет изменяться от 2 до 20.

\begin{figure}[H]
\centering
% График ошибки
\begin{tikzpicture}
\begin{axis}[
    xlabel={Число узлов},
    ylabel={Ошибка},
    grid=major,
    extra y ticks={0},
    tick style={line width=1pt},
    axis line style={line width=1.5pt},
]
\addplot[black, thick, mark=*, mark size=1.5pt] table[x index=0, y index=1] {
2 8.950449e-01
3 3.615968e-01
4 2.440653e-01
5 2.063839e-01
6 1.648331e-01
7 1.452169e-01
8 1.242140e-01
9 1.122282e-01
10 9.959451e-02
11 9.152292e-02
12 8.309768e-02
13 7.729564e-02
14 7.127979e-02
15 6.690948e-02
16 6.240016e-02
17 5.899046e-02
18 5.548535e-02
19 5.275122e-02
20 4.994878e-02
};
\end{axis}
\end{tikzpicture}
\caption{зависимость фактической ошибки от числа узлов}
\end{figure}

Как видно из полученного графика, фактическая ошибка имеет обратную степенную зависимость от числа узлов, и с увеличением числа узлов точность сначала резко возрастает, однако для последующего увелечения точности необходимо все большее число узлов. Однако, важно понимать, что слишком большое число узлов приведет к резкому снижению точности, поскольку полином получит большую степень.

\section{Выводы}

В ходе лабораторной работы было проведено исследование одного из методов интерполяции - полинома Ньютона. Были построены графики исходной функции и полученного полинома, ошибки в каждой точке построения, а также была исследована зависимость фактической ошибики от числа узлов и построен соответствующий график.

\end{document}

